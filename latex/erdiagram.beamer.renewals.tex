%
%  erdiagram.beamer.renewal.tex
%  ****************************
%
%  Renew macros defined in erdiagram so that 
%             hierarchical attributes
%             relational attribuites 
%             annotations identifying relationships ali=ongside above attribute names
%  appear successively as the presentation advances. 
%
% Created 22 Sep 2022.

% Slide 1 - shows logical model without scopes and without relationship ids
% Slide 2 - adds scopes -- these indicate equivalent paths i.e. commutative diagrams
% Slide 3 - reveals attributes required in a hierarchical implementation such as XML or Googles protocol buff IDL
% Silde 4 - reveals attributes required in a relational implementation
% slide 5 - annotates relationships with identifiers and tags attributes to
%            indicate the relationships which they are there to implement

%\erHierarchicalAttribute {#1 x} {#2 y} {#3 ismandatory}{#4 identifying} {#5 name} {#6 annotation}
\renewcommand {\erHierarchicalAttribute}[6]
{
\only<3-4>{\erRelorHierAttribute{#1}{#2}{#3}{#4}{#5}{}}
\only<5>{\erRelorHierAttribute{#1}{#2}{#3}{#4}{#5}{#6}}
}

%\erRelationalAttribute {#1 x} {#2 y} {#3 ismandatory}{#4 identifying} {#5 name} {#6 annotation}
\renewcommand {\erRelationalAttribute}[6]
{
\only<4>{\erRelorHierAttribute{#1}{#2}{#3}{#4}{#5}{}}
\only<5>{\erRelorHierAttribute{#1}{#2}{#3}{#4}{#5}{#6}}
}

% \errelid {#1 x} {#2 y} {#3 text anchor} {#4 text} 
\renewcommand {\errelid}[4]
{
\only<5>{\errelidbody{#1}{#2}{#3}{#4}}    
}

% \erscope {#1 x} {#2 y} {#3 text anchor} {#4 text}  15 March 2019
\renewcommand {\erscope}[4]
{
  \only<2>{\erscopebody{#1}{#2}{#3}{#4}}
}
