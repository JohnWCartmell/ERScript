
\documentclass[10pt,a4paper]{article}


\newenvironment{JCexample}
{\newcommand\colboxcolor{F0F0FF}%
\begin{lrbox}{\riddlebox}%
\begin{minipage}{\dimexpr\columnwidth-2\fboxsep\relax} \textbf{} \\ \itshape}
{\end{minipage}\end{lrbox}%
%\begin{center}
\colorbox[HTML]{\colboxcolor}{\usebox{\riddlebox}}
%\end{center}
}
\newcommand{\JCpictureFolder}[0]{../pictures}

%\JCplainFig{#1 pictureFilename}{#2 figureParam}{#3Caption}
\newcommand{\JCplainFig}[3]{
\begin{figure}[#2]
\caption{#3}
\label{#1}
\begin{center}
\input{\JCpictureFolder/#1}
\end{center}
\end{figure}
}

%\JCboxedFig{#1 pictureFilename}{#2 figureParam}{#3Caption}
\newcommand{\JCboxedFig}[3]{
\begin{figure}[#2]
\begin{JCexample}
\vspace{-0.5cm}
\caption{#3}
\label{#1}
\begin{center}
\input{\JCpictureFolder/#1}
\end{center}
\end{JCexample}
\end{figure}
}


%\JCbulletedFig{#1 pictureFilename}{#2 figureParam}{#3Caption}
\NewEnviron{JCbulletedFig}[3]{%
\setlength{\textfloatsep}{0pt}
\begin{figure}[#2]
\begin{JCexample}
\caption{#3}
\label{#1}
\begin{center}
$
\begin{array}{c m{0.25cm} | c m{6cm}}
\raisebox{-2.0cm}{
\input{\JCpictureFolder/#1}}& & & \text{\parbox{6cm}{\raggedright{\footnotesize{
\begin{enumerate}[(i)]
\BODY
\end{enumerate}}}}} \\
\end{array}
$
\end{center}
\end{JCexample}
\end{figure} 
}

\newlength{\JChalfHt}
%\JCbulletedDimFig{#1 pictureFilename}{#2 figureParam}{#3Caption}{#4PictureHeight}{#5TextWidth}
\NewEnviron{JCbulletedDimFig}[5]{%
\begin{figure}[#2]
\begin{JCexample}
\vspace{-0.5cm}
\caption{#3}
\label{#1}
\begin{center}
$
\begin{array}{c m{0.25cm} |  m{#5cm}}
\setlength{\JChalfHt}{#4cm * \real{0.5}}
\raisebox{-\JChalfHt}{
\input{\JCpictureFolder/#1}}& & \text{\parbox{#5cm}{\raggedright{\footnotesize{
\begin{enumerate}[(i)]
\BODY
\end{enumerate}}}}} \\
\end{array}
$
\end{center}
\end{JCexample}
\end{figure} 
}

\NewEnviron{JCnotedDimFig}[5]{%
\begin{figure}[#2]
\begin{JCexample}
\vspace{-0.5cm}
\caption{#3}
\label{#1}
\begin{center}
$
\begin{array}{c m{0.25cm} | c m{#5cm}}
\setlength{\JChalfHt}{#4cm * \real{0.5}}
\raisebox{-\JChalfHt}{
\input{\JCpictureFolder/#1}}& & & \text{\parbox{#5cm}{\raggedright{\footnotesize{
\BODY }}}}\\
\end{array}
$
\end{center}
\end{JCexample}
\end{figure} 
}


\newenvironment{JCquote}
{\begin{quote}\itshape}
{\end{quote}}


\newlength{\JCxmin}
\newlength{\JCxmid}
\newlength{\JCxmax}
\newlength{\JCymin}
\newlength{\JCymid}
\newlength{\JCymax}
\newlength{\JCxa}
\newlength{\JCxb}
\newlength{\JCxc}
\newlength{\JCxd}
\newlength{\JCxe}
\newlength{\JCxf}
\newlength{\JCxg}
\newlength{\JCxh}
\newlength{\JCxi}
\newlength{\JCya}
\newlength{\JCyb}
\newlength{\JCyc}
\newlength{\JCyd}
\newlength{\JCye}
\newlength{\JCyf}
\newlength{\JCyg}
\newlength{\JCyh}
\newlength{\JCyi}
\newlength{\JCabsoluteHeight}
\newlength{\JCabsoluteWidth}
\newlength{\JCabsypos}
\newlength{\JChangerx}
\newcommand{\JCsegments}[0]{1}
\definecolor{lightyellow}{cmyk}{0,0,0.3,0}
\newcommand{\JCEntityFillColour}[0]{white}
\newcommand{\JCToggleFillColour}[0]
{\ifthenelse{\equal{\JCEntityFillColour}{lightyellow}}{\renewcommand{\JCEntityFillColour}[0]{white}}
{\renewcommand{\JCEntityFillColour}[0]{lightyellow}}
}

%\entity{#1 width}{#2 height}{#3nodeprefix}{#4content}
\newcommand{\JCentity}[6]{
\JCToggleFillColour
\addtolength{\JCabsypos}{-#2cm}
\setlength{\JCxmin}{0cm}
\setlength{\JCymin}{0cm}
\setlength{\JCxmid}{#3cm * \real{0.5}}
\setlength{\JCymid}{#4cm * \real{0.5}}
\setlength{\JCxmax}{#3cm}
\setlength{\JCymax}{#4cm}
\setlength{\JCxa}{#3cm * \real{0.1}}
\setlength{\JCxb}{#3cm * \real{0.2}}
\setlength{\JCxc}{#3cm * \real{0.3}}
\setlength{\JCxd}{#3cm * \real{0.4}}
\setlength{\JCxe}{#3cm * \real{0.5}}
\setlength{\JCxf}{#3cm * \real{0.6}}
\setlength{\JCxg}{#3cm * \real{0.7}}
\setlength{\JCxh}{#3cm * \real{0.8}}
\setlength{\JCxi}{#3cm * \real{0.9}}
\setlength{\JCya}{#4cm * \real{0.1}}
\setlength{\JCyb}{#4cm * \real{0.2}}
\setlength{\JCyc}{#4cm * \real{0.3}}
\setlength{\JCyd}{#4cm * \real{0.4}}
\setlength{\JCye}{#4cm * \real{0.5}}
\setlength{\JCyf}{#4cm * \real{0.6}}
\setlength{\JCyg}{#4cm * \real{0.7}}
\setlength{\JCyh}{#4cm * \real{0.8}}
\setlength{\JCyi}{#4cm * \real{0.9}}
\rput[bl](#1,#2){\pspicture(#3,#4)
%\psgrid
\rput[bl](0,0){\psframe[framearc=.2,fillstyle=solid,fillcolor=\JCEntityFillColour](0,0)(#3,#4)}
\rput[bl](0,0){#6}
\pnode(\JCxmax,\JCymid){#5R}
\pnode(\JCxmin,\JCymid){#5L}
\pnode(\JCxmid,\JCymin){#5B}
\pnode(\JCxmid,\JCymax){#5T}
\pnode(\JCxa,\JCymax){#5Ta}
\pnode(\JCxb,\JCymax){#5Tb}
\pnode(\JCxc,\JCymax){#5Tc}
\pnode(\JCxd,\JCymax){#5Td}
\pnode(\JCxe,\JCymax){#5Te}
\pnode(\JCxf,\JCymax){#5Tf}
\pnode(\JCxg,\JCymax){#5Tg}
\pnode(\JCxh,\JCymax){#5Th}
\pnode(\JCxi,\JCymax){#5Ti}
\pnode(\JCxa,\JCymin){#5Ba}
\pnode(\JCxb,\JCymin){#5Bb}
\pnode(\JCxc,\JCymin){#5Bc}
\pnode(\JCxd,\JCymin){#5Bd}
\pnode(\JCxe,\JCymin){#5Be}
\pnode(\JCxf,\JCymin){#5Bf}
\pnode(\JCxg,\JCymin){#5Bg}
\pnode(\JCxh,\JCymin){#5Bh}
\pnode(\JCxi,\JCymin){#5Bi}
\pnode(\JCxmin,\JCya){#5La}
\pnode(\JCxmin,\JCyb){#5Lb}
\pnode(\JCxmin,\JCyc){#5Lc}
\pnode(\JCxmin,\JCyd){#5Ld}
\pnode(\JCxmin,\JCye){#5Le}
\pnode(\JCxmin,\JCyf){#5Lf}
\pnode(\JCxmin,\JCyg){#5Lg}
\pnode(\JCxmin,\JCyh){#5Lh}
\pnode(\JCxmin,\JCyi){#5Li}
\pnode(\JCxmax,\JCya){#5Ra}
\pnode(\JCxmax,\JCyb){#5Rb}
\pnode(\JCxmax,\JCyc){#5Rc}
\pnode(\JCxmax,\JCyd){#5Rd}
\pnode(\JCxmax,\JCye){#5Re}
\pnode(\JCxmax,\JCyf){#5Rf}
\pnode(\JCxmax,\JCyg){#5Rg}
\pnode(\JCxmax,\JCyh){#5Rh}
\pnode(\JCxmax,\JCyi){#5Ri}
\endpspicture
}
\setlength{\JChangerx}{#1cm + (#3cm * \real{0.5})}
\addtolength{\JCabsypos}{#2cm}
\pnode(\JChangerx,\JCabsypos){#5hanger}
\JCToggleFillColour
}

\newcommand {\JCabsolute}[4]{
\setlength{\JCabsoluteHeight}{#4cm * \real{0.5}}
\setlength{\JCabsoluteWidth}{#3cm * \real{2}}
\rput[bl](#1,#2){\pspicture*[dimen=outer,border=0](0,0)(20,\JCabsoluteHeight)
\JCentity{0}{0}{#3}{#4}{absolute}{}
\endpspicture}
}

\newcommand {\JCemstart}[2]
{\pspicture(#1,#2)
%\psgrid
\setlength{\JCabsypos}{#2cm-0.25cm}
\JCabsolute{0}{\JCabsypos}{#1}{0.5}
}
\newcommand{\JCemend}{\endpspicture}



\newcommand \JCsimple[6]{
  \JCentity{#1}{#2}{#3}{#4}{#5}{\rput[B](\JCxe,\JCyd){\footnotesize \textit{#6}}}
  }

\newcommand{\JCsetangleA}[1]{
\reversestring[q]{#1}
\substring[q]{\thestring}{1}{2}
\whereisword[q]{\thestring}{L}
\ifthenelse{\equal{\theresult}{0}}
{\renewcommand{\JCrelangleA}{0}}
{\renewcommand{\JCrelangleA}{180}}
%\stringlength[q]{#1}
\substring[q]{#1}{$}{$}
\whereisword[q]{abcd}{\thestring}
\ifthenelse{\equal{\theresult}{0}}
{\renewcommand{\JCrelpos}{efghi}}
{\renewcommand{\JCrelpos}{abcd}}
}

\newcommand{\JCsetangleB}[1]{
\reversestring[q]{#1}
\substring[q]{\thestring}{1}{2}
\whereisword[q]{\thestring}{L}
\ifthenelse{\equal{\theresult}{0}}
{\renewcommand{\JCrelangleB}{0}}
{\renewcommand{\JCrelangleB}{180}}
}

% 3/05/2012
%\newcommand {\JCrelarmlen} {1.5} before 11/10/2012
\newcommand {\JCrelarmlen} {1.0} % after 15/10/2012
\newcommand {\JCrellinearc} {0.10}
\newcommand {\JCrelcrowlen} {0.2}
\newcommand {\JCrelcrowwidth} {0.1}
\newcommand {\JCrelcrowarc} {0.2}
\newcommand {\JCrelcrowbendx}{.05}
\newcommand {\JCrelcrowbendy}{0.17}
\newcommand {\JCrelangleA}{0}
\newcommand {\JCrelangleB}{180}
\newcommand {\JCrelpos}{abcd}

\newcommand{\JCrel}[4]{
\JCsetangleA{#1}
\JCsetangleB{#2}
\ncdiag[angleA=\JCrelangleA, angleB=\JCrelangleB,arm=\JCrelarmlen, linearc=\JCrellinearc,linestyle=dashed,dash=3pt 2pt]{-}{#1}{#2}
\ifthenelse{\equal{\JCrelangleA}{0}}{
\ifthenelse{\equal{\JCrelpos}{abcd}}{
\nbput[npos=0,refangle=dr,labelsep=0.1]{\rput[tl](0,0){\footnotesize{\textit{#4}}}}
}{
\naput[npos=0,refangle=ur,labelsep=0.1]{\rput[bl](0,0){\footnotesize{\textit{#4}}}}
}
}
{
\ifthenelse{\equal{\JCrelpos}{abcd}}{
\naput[npos=0,refangle=ur,labelsep=0.1]{\rput[tr](0,0){\footnotesize{\textit{#4}}}}
}
{
\nbput[npos=0,refangle=dr,labelsep=0.1]{\rput[br](0,0){\footnotesize{\textit{#4}}}}
}
}
\ncput{\pnode{xxx}}
\renewcommand{\JCsegments}[0]{3}
\ifthenelse{\isin{Amany}{#3}}{
% left hand end
\ncput[nrot=:R,npos=0]{\psline[linearc=\JCrelcrowarc](0,\JCrelcrowlen)(\JCrelcrowbendx,\JCrelcrowbendy)(\JCrelcrowwidth,0)}
\ncput[nrot=:R,npos=0]{\psline[linearc=\JCrelcrowarc](0,\JCrelcrowlen)(-\JCrelcrowbendx,\JCrelcrowbendy)(-\JCrelcrowwidth,0)}
\ncput[nrot=:R,npos=0]{\psline(0,\JCrelcrowlen)(0,0)}
}{}
\ifthenelse{\isin{Bmany}{#3}}{
% Right hand end
\ncput[nrot=:R,npos=\JCsegments]{\psline[linearc=\JCrelcrowarc](0,-\JCrelcrowlen)(\JCrelcrowbendx,-\JCrelcrowbendy)(\JCrelcrowwidth,0)}
\ncput[nrot=:R,npos=\JCsegments]{\psline[linearc=\JCrelcrowarc](0,-\JCrelcrowlen)(-\JCrelcrowbendx,-\JCrelcrowbendy)(-\JCrelcrowwidth,0)}
\ncput[nrot=:R,npos=\JCsegments]{\psline(0,-\JCrelcrowlen)(0,0)}
}{}
\ifthenelse{\isin{Amand}{#3}}{
\ncdiag[angleA=\JCrelangleA, angleB=\JCrelangleB,armA=\JCrelarmlen, armB=0,linearc=\JCrellinearc]{-}{#1}{xxx}
}{}
\ifthenelse{\isin{Bmand}{#3}}{
\ncdiag[angleA=\JCrelangleA, angleB=\JCrelangleB,armA=0, armB=\JCrelarmlen,linearc=\JCrellinearc]{-}{xxx}{#2}
}{}
}

\newcommand {\JCcomparmlen} {0.25}
\newcommand {\JCreccarmlen} {0.45}

\newcommand{\JCcomp}[3]{
    \ifthenelse{\isin{clockwise}{#3}}{
        \ncloop[angleA=-90, angleB=90,loopsize=1.3,armA=\JCreccarmlen,armB=\JCreccarmlen,linearc=.1,linestyle=dashed,dash=3pt 2pt ]{-}{#1}{#2}
        \renewcommand{\JCsegments}[0]{5}
    }
    {
        \ncdiag[angleA=-90, angleB=90,arm=\JCcomparmlen, linearc=\JCrellinearc,linestyle=dashed,dash=3pt 2pt ]{-}{#1}{#2}
        \renewcommand{\JCsegments}[0]{3}
    }
    \ncput{\pnode{xxx}}


  \ifthenelse{\isin{many}{#3}}{
    % Right hand end
    \ncput[nrot=:R,npos=\JCsegments]{\psline[linearc=\JCrelcrowarc](0,-\JCrelcrowlen)(\JCrelcrowbendx,-\JCrelcrowbendy)(\JCrelcrowwidth,0)}
    \ncput[nrot=:R,npos=\JCsegments]{\psline[linearc=\JCrelcrowarc](0,-\JCrelcrowlen)(-\JCrelcrowbendx,-\JCrelcrowbendy)(-\JCrelcrowwidth,0)}
    \ncput[nrot=:R,npos=\JCsegments]{\psline(0,-\JCrelcrowlen)(0,0)}
  }{}
  \ifthenelse{ \not \isin{opt}{#3}}{
      \ifthenelse{\isin{clockwise}{#3}}{
          \ncangle[angleA=-90, angleB=90,armA=0, armB=\JCreccarmlen,linearc=\JCrellinearc]{-}{xxx}{#2}
      }
      {     
          \ncdiag[angleA=-90, angleB=90,armA=0, armB=\JCcomparmlen,linearc=\JCrellinearc]{-}{xxx}{#2}
      }
  }{}
  \ifthenelse{\isin{mand}{#3} }{
      \ifthenelse{\isin{clockwise}{#3}}{
          \ncangles[angleA=-90, angleB=90,armA=\JCreccarmlen, armB=0,linearc=\JCrellinearc]{-}{#1}{xxx}
      }
      {
      
          \ncdiag[angleA=-90, angleB=90,armA=\JCcomparmlen, armB=0,linearc=\JCrellinearc]{-}{#1}{xxx}
      }
   }{}
   % repeat the whole line so context esists for labels
    \ifthenelse{\isin{clockwise}{#3}}{
        \ncloop[angleA=-90, angleB=90,loopsize=1.3,armA=\JCreccarmlen,armB=\JCreccarmlen,linearc=.1,linestyle=dashed,dash=3pt 2pt ]{-}{#1}{#2}
        \renewcommand{\JCsegments}[0]{5}
    }
    {
        \ncdiag[angleA=-90, angleB=90,arm=\JCcomparmlen, linearc=\JCrellinearc,linestyle=dashed,dash=3pt 2pt ]{-}{#1}{#2}
        \renewcommand{\JCsegments}[0]{3}
    }
}


\newlength{\JCattrTop}
\newcommand{\JCattributesThemselves}[2]{
%\fbox{
\begin{minipage}{#1}
\setlist{noitemsep}
\textit{\footnotesize{
\begin{itemize}[label=$\succ$,itemindent=-0.5cm,labelsep=0cm,leftmargin=0.5cm,rightmargin=0cm]
#2
\end{itemize}
}}
\end{minipage}
%}
}

\newcommand{\JCattributes}[2]{
\setlength{\JCattrTop}{\JCymax}
\addtolength{\JCattrTop}{-0.55cm}
\rput[tr](\JCxmax,\JCattrTop){\JCattributesThemselves{#1}{#2}}
}

\newcommand{\JCgattributes}[2]{
\setlength{\JCattrTop}{\JCyg}
\addtolength{\JCattrTop}{-0.35cm}
\rput[tr](\JCxmax,\JCattrTop){\JCattributesThemselves{#1}{#2}}
}

\newcommand{\JCctAbstractEntityName}[1]{
\uput[d](\JCxmid,\JCymax){\underline{\textit{#1}}}
}

\newcommand{\JCctEntityName}[1]{
\uput[d](\JCxmid,\JCymax){\textit{#1}}
}

\newcommand{\JCsimpleEntityName}[1]{
\rput(\JCxmid,\JCymid){
\begin{minipage}
{\JCxmax}{\begin{center}\textit{#1}\end{center}}
\end{minipage}
}
}

\newcommand{\JCgEntityName}[1]{
\rput(\JCxmid,\JCyg){
\begin{minipage}
{\JCxmax}{\begin{center}\textit{#1}\end{center}}
\end{minipage}
}
}
\author{ John Cartmell}
\title{Entity Model}
%\author{John Cartmell}
\renewcommand{\JCpictureFolder}[0]{../chapter_1_perspective/pictures}
\begin{document}
\maketitle
\section {JCplainFig command}

\begin{verbatim}
\JCplainFig{pictureFilename}{floatDirective}{Caption}
\end{verbatim}

\JCplainFig{picture1}{h}{Picture one from folder of chapter one}

\section {JCboxedFig command}

\begin{verbatim}
\JCboxedFig{pictureFilename}{floatDirective}{Caption}
\end{verbatim}

\JCboxedFig{picture1}{h}{Picture one from folder of chapter one}

\section {JCbulletedFig environment}

\begin{verbatim}
\begin{\JCbulletedFig}{pictureFilename}{floatDirective}{Caption}
\item{text}
\item{text}
etc.
\end{\JCbulletedFig}
\end{verbatim}


\begin{JCbulletedFig}{picture1}{h}{Picture one from folder of chapter one with bullets}
\item{item 1}
\item{item 2}
\end{JCbulletedFig}

\section{Notation}

%\psset{unit=.9cm}
\subsection {JCsimple}
A macro for the simplest entity type - one with no subtypes just a name:

\begin{verbatim}
\JCsimple{bottom left x pos}{bottom left y pos}{width}{height}{entitytypename}{EntityTypeName}
\end{verbatim}
\pspicture(14,3)
\JCsimple{5.5}{0}{3.5}{1}{entitytypename}{EntityTypeName}
\endpspicture

\subsection {entity model}

\JCemstart{14}{3}
\JCsimple{1.5}{0}{2.5}{1}{a}{Entity A}
\JCcomp{ahanger}{aT}{many}
\JCemend

\subsection {JCrel}
Join two entiity types by a reference relationship:
\begin{verbatim}
\JCrel{entitytypename1<label>}{entitytypename2<label>}{cardinality indication}{rolename of relationship}
\end{verbatim}
For example
\begin{verbatim}
\JCsimple{1.5}{0}{2.5}{1}{a}{Entity A}
\JCsimple{7.5}{0}{2.5}{1}{b}{Entity B}
\JCrel{aR}{bL}{AmanyBmany}{associated with}
\end{verbatim}
\pspicture(14,3)
\JCsimple{1.5}{0}{2.5}{1}{a}{Entity A}
\JCsimple{7.5}{0}{2.5}{1}{b}{Entity B}
\JCrel{aR}{bL}{AmanyBmany}{associated with}
\endpspicture

\subsection {JCrel}
Change the points of attachment of reference relationships
\begin{verbatim}
\JCrel{entitytypename1<label>}{entitytypename2<label>}{cardinality indication}{rolename of relationship}
\end{verbatim}
For example
\begin{verbatim}
\JCsimple{1.5}{2}{2.5}{1}{a}{Entity A}
\JCsimple{7.5}{0}{2.5}{1}{b}{Entity B}
\JCsimple{7.5}{4}{2.5}{1}{c}{Entity C}
\JCrel{aRb}{bL}{AmanyBmany}{\parbox{2cm}{associated with}}
\JCrel{aRg}{cL}{AmanyBmany}{associated with}
\end{verbatim}
\pspicture(14,5)
\JCsimple{1.5}{2}{2.5}{1}{a}{Entity A}
\JCsimple{7.5}{0}{2.5}{1}{b}{Entity B}
\JCsimple{7.5}{4}{2.5}{1}{c}{Entity C}
\JCrel{aRb}{bL}{AmanyBmany}{\parbox{1.5cm}{associated with}}
\JCrel{aRg}{cL}{AmanyBmany}{associated with}
\endpspicture

\subsection{Entity Type Nesting}
\vspace{1cm}
\pspicture(14,7.5)
%\psgrid
%\rput[bl](0.0,0){\psaxes(14,7.5)}
\JCentity{0}{0}{9}{7.5}{particle}
{
  \rput[bl](0.5,6.9){\underline{Particle}}
  \JCentity{0.5}{0.5}{8}{3.2}{fermiparticle}
  {  
     \rput[bl](0.4,2.2){\underline{Fermi Particle}}
     \JCsimple{0.5}{0.4}{2}{1}{nucleon}{\underline{Nucleon}}
     \JCsimple{3.0}{0.4}{2}{1}{neutrino}{\underline{Neutrino}}
     \JCsimple{5.5}{0.4}{2}{1}{baryon}{\underline{Baryon}}
     \JCsimple{3.0}{1.8}{2}{1}{electron}{electron}
     \JCsimple{5.5}{1.8}{2}{1}{muon}{muon}
  }
  \JCentity{0.5}{4.3}{8}{2.3}{boseparticle}
  {
    \rput[bl](0.4,1.6){\underline{Bose Particle}}
    \JCsimple{0.5}{0.4}{2}{1}{photon}{photon}
    \JCsimple{3.0}{0.4}{2}{1}{meson}{\underline{Meson}}
    \JCsimple{5.5}{0.4}{2}{1}{graviton}{graviton}
  }
}
\endpspicture

\subsection{The absolute}
\vspace{1cm}
\pspicture[dimen=middle,border=0](14,2)
\psgrid
\JCabsolute{0}{1}{10}{0.5}
\endpspicture
\vspace{1cm}


\section{Scope of relationships}
Most models have constraints on and between relationships and though they are significant they not expressed in the model. 
Constraints on the scope of a relationship are one kind of such constraint and they are the most important for reasons that we explain.
The concept of the scope of a relationship is a missing concept in entity modelling.

\subsection {Scope Squares}
Example:
\vspace{1cm}
\begin{center}
\JCemstart{8.5}{6}
%\psaxes(8.5,6)
\JCsimple{6}{3}{2.3}{1.3}{play}{play}
\JCsimple{6}{0.5}{2.3}{1.3}{ char}{character}
\JCsimple{0.5}{3}{2.3}{1.3}{performance}{performance}
\JCsimple{0.5}{0.5}{2.3}{1.3}{actor}{cast member}
\JCcomp{playhanger}{playT}{many mand}
\JCcomp{playB}{charT}{many mand}
\JCcomp{performanceB}{actorT}{many mand}
\JCcomp{performancehanger}{performanceT}{many mand}
\JCrel{performanceR}{playL}{Amany Amand}{\parbox{1.3cm}{performance of}}
\JCrel{actorR}{charL}{Amany Amand}{\parbox{1.1cm}{plays part of}}
\JCemend
\end{center}
\vspace{1cm}



\section{example}
Opening text...

\vspace{1cm}

\JCemstart{14}{9}
%\psaxes(14,9)
\rput(10,7){10101text}
\JCentity{1}{6}{2}{1}{task}{\rput[bl](0.25,0.4){task}}
%\ncloop[angleA=-90, angleB=90,loopsize=1.3,arm=.5,linearc=.1]{->}{taskBc}{taskTc}
\JCcomp{taskBc}{taskTc}{clockwise opt mand many}
\JCentity{6.5}{3}{2}{1.5}{analyte}{\rput[bl](0.25,0.4){analyte}}
\JCentity{0}{1}{2}{3}{test}{\rput[bl](0.25,0.25){test}}
\ncloop[angleB=180,loopsize=-1,arm=.5,linearc=.1]{->}{testRa}{testLa}
\JCentity{4}{0}{2.5}{1}{compoundMap}{compoundMap}
\ncloop[angleA=90, angleB=-90,loopsize=2,arm=.5,linearc=.1]{<-}{compoundMapT}{compoundMapB}
\ncloop[angleA=-90, angleB=90,loopsize=-1,arm=.5,linearc=.1]{->}{analyteBi}{taskTg}
\JCentity{10}{0.5}{2.5}{2}{c}{\rput[bl](0.25,0.4){compound}}
\ncloop[angleA=180,loopsize=1,arm=1,linearc=.1]{->}{testLb}{cRb}
\ncloop[angleA=180,loopsize=2,arm=1.5,linearc=.1]{->}{cLe}{cRe}
\JCcomp{taskhanger}{taskT}{many}
\JCcomp{changer}{cT}{many}
\JCcomp{analytehanger}{analyteT}{many mand}
\JCcomp{taskB}{testT}{many}
\JCrel{testRh}{analyteL}{AmanyBmany}{\parbox{1.5cm}{analytic subject}}
\JCrel{testRf}{cLf}{Amany}{subject dosed}
\JCrel{testRb}{compoundMapL}{Amany}{dosed from}
\JCemend
\vspace{1cm}
\subsection{Spare}
\pspicture(-3,0)(10,-5)
%\psgrid
\JCentity{2.5}{-1}{1.5}{.6}{root}
{
\rput[B](0.75,.1){whole}
\JCToggleFillColour
\JCentity{-4}{-2}{1.5}{.6}{P1}{\rput[B](0.75,.1){P1}
\JCToggleFillColour
\JCentity{-.875}{-2}{.75}{.6}{P11}{\rput[B](0.35,.15){P11}}
\JCentity{0.125}{-2}{.75}{.6}{P12}{\rput[B](0.35,.15){P12}
}
\qdisk(1.075,-1.7){1pt}
\qdisk(1.25,-1.7){1pt}
\qdisk(1.425,-1.7){1pt}
\JCentity{1.625}{-2}{.75}{.6}{P13}{\rput[B](0.35,.15){P1n}}
}
\JCentity{-0.5}{-2}{1.5}{.6}{P2}{\rput[B](0.75,.1){P2}
\JCToggleFillColour
\JCentity{-.875}{-2}{.75}{.6}{P21}{\rput[B](0.35,.15){P21}}
\JCentity{0.125}{-2}{.75}{.6}{P22}{\rput[B](0.35,.15){P22}
}
\qdisk(1.075,-1.7){1pt}
\qdisk(1.25,-1.7){1pt}
\qdisk(1.425,-1.7){1pt}
\JCentity{1.625}{-2}{.75}{.6}{P23}{\rput[B](0.35,.15){P2n}}
}
\qdisk(2.4,-1.7){1pt}
\qdisk(2.7,-1.7){1pt}
\qdisk(2,-1.7){1pt}
\JCentity{3.5}{-2}{1.5}{.6}{Pn}{\rput[B](0.75,.1){Pn}
\JCToggleFillColour
\JCentity{-.875}{-2}{.75}{.6}{Pn1}{\rput[B](0.35,.15){Pn1}}
\JCentity{0.125}{-2}{.75}{.6}{Pn2}{\rput[B](0.35,.15){Pn2}
}
\qdisk(1.075,-1.7){1pt}
\qdisk(1.25,-1.7){1pt}
\qdisk(1.425,-1.7){1pt}
\JCentity{1.625}{-2}{.75}{.6}{Pnm}{\rput[B](0.35,.15){Pnm}}
}
}
\JCcomp{rootBc}{P1T}{many mand}
\JCcomp{rootBe}{P2T}{many mand}
\JCcomp{rootBg}{PnT}{many mand}
\JCcomp{P1Bc}{P11T}{many mand}
\JCcomp{P1Be}{P12T}{many mand}
\JCcomp{P1Bg}{P13T}{many mand}
\JCcomp{P2Bc}{P21T}{many mand}
\JCcomp{P2Be}{P22T}{many mand}
\JCcomp{P2Bg}{P23T}{many mand}
\JCcomp{PnBc}{Pn1T}{many mand}
\JCcomp{PnBe}{Pn2T}{many mand}
\JCcomp{PnBg}{PnmT}{many mand}
\endpspicture

\vspace{0.5cm}
\begin{JCexample}
\noindent In some story book (I ask you to believe), every creature has a head, one or more segments and a tail:
\begin{center}
\raisebox{-1cm}{
\pspicture(-1.85,1.4)(3.25,4)
%\psgrid
\JCentity{0}{3}{1.5}{0.6}{whole}{\rput[B](0.75,0.15){\footnotesize{\textit{creature}}}}
\JCentity{-1.85}{1.5}{1.6}{0.6}{partA}{\rput[B](0.8,0.15){\footnotesize{\textit{head}}}}
\JCentity{-0.05}{1.5}{1.6}{0.6}{partB}{\rput[B](0.8,0.15){\footnotesize{\textit{segment}}}}
\JCentity{1.85}{1.5}{1.6}{0.6}{partC}{\rput[B](0.8,0.15){\footnotesize{\textit{tail}}}}
\JCcomp{wholeBc}{partAT}{mand}
\JCcomp{wholeBe}{partBT}{many mand}
\JCcomp{wholeBg}{partCT}{mand}
\endpspicture}
\end{center}
\end{JCexample}


\subsection{Note on Notation}
The notations that are being used are those of Richard Barker and SSADM - except that that notation does not distinguish composition relationships from any others.
The idea of distinguishing composition relationships and the term itself comes from PCTE. The idea of top-down modelling of composition relationships in
entity relationships diagrams is inspired by older notations - very much as captured in the phrases 'top-down' and 'bottom-up'.

\subsection{Cardinality of the Composition Relationship}
\vspace{.4cm}
\noindent If there can be no more than one of a particular type of part in any given whole then the relationship is shown without the crows foot; 
if in some cases there may be no parts of a particular type then the relationship line is shown dashed. This gives four possibilities for the cardinality of the composition relationship:
\begin{center}
$
\begin{array}{c p{0.5cm} | p{0.5cm} c}
\pspicture(-0.9,1.4)(0.6,4)
%\psgrid
\JCentity{-1}{3}{1.5}{0.6}{whole}{\rput[B](0.75,0.15){\footnotesize{\textit{whole type}}}}
\JCentity{-1}{1.5}{1.5}{0.6}{part}{\rput[B](0.75,0.15){\footnotesize{\textit{part type}}}}
\JCcomp{wholeB}{partT}{}
\endpspicture & & &
\pspicture(-0.9,1.4)(0.6,4)
%\psgrid
\JCentity{-1}{3}{1.5}{0.6}{whole}{\rput[B](0.75,0.15){\footnotesize{\textit{whole type}}}}
\JCentity{-1}{1.5}{1.5}{0.6}{part}{\rput[B](0.75,0.15){\footnotesize{\textit{part type}}}}
\JCcomp{wholeB}{partT}{mand}
\endpspicture \\
\makebox[3cm]{zero or one} & & & \makebox[3cm]{exactly one} \\
\hline
\pspicture(-0.9,1.4)(0.6,4)
%\psgrid
\JCentity{-1}{3}{1.5}{0.6}{whole}{\rput[B](0.75,0.15){\footnotesize{\textit{whole type}}}}
\JCentity{-1}{1.5}{1.5}{0.6}{part}{\rput[B](0.75,0.15){\footnotesize{\textit{part type}}}}
\JCcomp{wholeB}{partT}{many}
\endpspicture & & &
\pspicture(-0.9,1.4)(0.6,4)
%\psgrid
\JCentity{-1}{3}{1.5}{0.6}{whole}{\rput[B](0.75,0.15){\footnotesize{\textit{whole type}}}}
\JCentity{-1}{1.5}{1.5}{0.6}{part}{\rput[B](0.75,0.15){\footnotesize{\textit{part type}}}}
\JCcomp{wholeB}{partT}{mand many}
\endpspicture \\
\makebox[3cm]{zero,one or more} & & & \makebox[3cm]{one or more}
\end{array}
$
\end{center}
\vspace{.4cm}
\begin{center}
\raisebox{-2.75cm}{
\pspicture(-1,-5.5)(6,0)
%\psgrid
\JCentity{2.5}{-1}{1.5}{.6}{root}
{
\rput[B](0.75,.1){\footnotesize{\textit{whole type}}}
\JCToggleFillColour
\JCentity{-2.5}{-2}{1.7}{.6}{P1}{\rput[B](.85,.1){\footnotesize{\textit{part type 1}}}}
\JCentity{-.5}{-2}{1.7}{.6}{P2}{\rput[B](.85,.1){\footnotesize{\textit{part type 2}}}
\JCToggleFillColour
\JCentity{-2.5}{-2}{2}{.6}{P21}{\rput[B](1,.1){\footnotesize{\textit{subpart type 1}}}}
\JCentity{-.3}{-2}{2}{.6}{P22}{\rput[B](1,.1){\footnotesize{\textit{subpart type 2}}}}
\JCentity{1.8}{-2}{2}{.6}{P23}{\rput[B](1,.1){\footnotesize{\textit{subpart type 3}}}}
}
\JCentity{1.6}{-2}{1.7}{.6}{P3}{\rput[B](.85,.1){\footnotesize{\textit{part type 3}}}}
}
\JCcomp{rootBc}{P1T}{many mand}
\JCcomp{rootBe}{P2T}{many mand}
\JCcomp{rootBg}{P3T}{many mand}
\JCcomp{P2Bc}{P21T}{many mand}
\JCcomp{P2Be}{P22T}{many mand}
\JCcomp{P2Bg}{P23T}{many mand}
\endpspicture}
\end{center}



\subsection {Hierarchy}
\vspace{.5cm}
\noindent If there are multiple types of part each composed of multiple types of subpart we get a three level tree structure of types and relationships like this:
\begin{center}
\pspicture(-3,-5.5)(10,0)
%\psgrid
\JCentity{2.5}{-1}{1.5}{.6}{root}
{
\rput[B](0.75,.1){}
\JCToggleFillColour
\JCentity{-4}{-2}{1.5}{.6}{P1}{\rput[B](0.75,.1){}
\JCToggleFillColour
\JCentity{-.875}{-2}{.75}{.6}{P11}{\rput[B](0.35,.15){}}
\JCentity{0.125}{-2}{.75}{.6}{P12}{\rput[B](0.35,.15){}
}
\qdisk(1.075,-1.7){1pt}
\qdisk(1.25,-1.7){1pt}
\qdisk(1.425,-1.7){1pt}
\JCentity{1.625}{-2}{.75}{.6}{P13}{\rput[B](0.35,.15){}}
}
\JCentity{-0.5}{-2}{1.5}{.6}{P2}{\rput[B](0.75,.1){}
\JCToggleFillColour
\JCentity{-.875}{-2}{.75}{.6}{P21}{\rput[B](0.35,.15){}}
\JCentity{0.125}{-2}{.75}{.6}{P22}{\rput[B](0.35,.15){}
}
\qdisk(1.075,-1.7){1pt}
\qdisk(1.25,-1.7){1pt}
\qdisk(1.425,-1.7){1pt}
\JCentity{1.625}{-2}{.75}{.6}{P23}{\rput[B](0.35,.15){}}
}
\qdisk(2.4,-1.7){1pt}
\qdisk(2.7,-1.7){1pt}
\qdisk(2,-1.7){1pt}
\JCentity{3.5}{-2}{1.5}{.6}{Pn}{\rput[B](0.75,.1){}
\JCToggleFillColour
\JCentity{-.875}{-2}{.75}{.6}{Pn1}{\rput[B](0.35,.15){}}
\JCentity{0.125}{-2}{.75}{.6}{Pn2}{\rput[B](0.35,.15){}
}
\qdisk(1.075,-1.7){1pt}
\qdisk(1.25,-1.7){1pt}
\qdisk(1.425,-1.7){1pt}
\JCentity{1.625}{-2}{.75}{.6}{Pnm}{\rput[B](0.35,.15){}}
}
}
\JCcomp{rootBc}{P1T}{many mand}
\JCcomp{rootBe}{P2T}{many mand}
\JCcomp{rootBg}{PnT}{many mand}
\JCcomp{P1Bc}{P11T}{many mand}
\JCcomp{P1Be}{P12T}{many mand}
\JCcomp{P1Bg}{P13T}{many mand}
\JCcomp{P2Bc}{P21T}{many mand}
\JCcomp{P2Be}{P22T}{many mand}
\JCcomp{P2Bg}{P23T}{many mand}
\JCcomp{PnBc}{Pn1T}{many mand}
\JCcomp{PnBe}{Pn2T}{many mand}
\JCcomp{PnBg}{PnmT}{many mand}
\endpspicture
\end{center}

\noindent The nodes of such a tree are entity types and the branches composition relationships. 

\vspace{3cm}
\subsection{Relativisation or Localisation}
Comma categories!

\subsection{Spare}
\begin{JCexample}
\begin{center}
$
\begin{array}{c m{1cm} | c m{6cm}}
\raisebox{-1cm}{
\pspicture(-0.9,1.4)(0.6,4)
%\psgrid
\JCentity{-1}{3}{1.5}{0.6}{sentence}{\rput[B](0.75,0.15){\footnotesize{\textit{coin}}}}
\JCentity{-1}{1.5}{1.5}{0.6}{noun}{\rput[B](0.75,0.15){\footnotesize{\textit{face}}}}
\JCcomp{sentenceBc}{nounTc}{mand opt}
\nbput[npos=0,ref=tr,labelsep=0.1]{\rput[Br](0,-0.25){\footnotesize heads}}
\JCcomp{sentenceBg}{nounTg}{mand opt}
\naput[npos=0,ref=tl,labelsep=0.1]{\rput[Bl](0,-0.25){\footnotesize tails}}
\psellipticarc[showpoints=false]{-}(-0.25,1.9)(.8,.5){30}{150}
\endpspicture} & & & \text{\parbox{6cm}{\raggedright{\footnotesize{
\begin{itemize}
\item{every \underline{coin} has a \underline{heads} which is a \underline{face}}
\item{every \underline{coin} has an \underline{tails} which is a \underline{face}}
\end{itemize} }}}} \\
\end{array}
$
\end{center}
\end{JCexample}

\noindent Two or more of the part types might have a common abstraction, as for example here:
\raisebox{-1cm}{\raisebox{1cm}{(d)}
\pspicture(-2.3,1.5)(1.8,4)
%\psgrid
\JCentity{-1.05}{3}{1.5}{0.6}{whole}{\rput[B](0.75,0.15){\footnotesize{\textit{whole type}}}}
\JCentity{-2.25}{1.4}{3.9}{1}{part}{
\JCentity{0.25}{0.2}{1.6}{0.6}{partA}{\rput[B](0.8,0.15){\footnotesize{\textit{part type 1}}}}
\JCentity{2.1}{0.2}{1.6}{0.6}{partB}{\rput[B](0.8,0.15){\footnotesize{\textit{part type 2}}}}
}
\JCcomp{wholeBe}{partTe}{many mand}
\endpspicture}
or here:
\raisebox{-1.65cm}{
\raisebox{1.65cm}{(e)}
\pspicture(-2.3,.8)(3,3.6)
%\psgrid
\JCentity{0}{3}{1.7}{0.6}{whole}{\rput[B](0.8,0.15){\footnotesize{\textit{whole type}}}}
\JCentity{-2.25}{1.4}{3.9}{1}{part}{
\JCentity{0.25}{0.2}{1.6}{0.6}{partA}{\rput[B](0.8,0.15){\footnotesize{\textit{part type 1}}}}
\JCentity{2.1}{0.2}{1.6}{0.6}{partB}{\rput[B](0.8,0.15){\footnotesize{\textit{part type 2}}}}
}
\JCentity{1.85}{1.65}{1.6}{0.6}{partC}{\rput[B](0.8,0.15){\footnotesize{\textit{part type 3}}}}
\JCcomp{wholeBb}{partTe}{many mand}
\JCcomp{wholeBh}{partCT}{many mand}
\endpspicture}
.



\end{document}
